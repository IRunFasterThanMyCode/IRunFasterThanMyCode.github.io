%%%%%%%%%%%%%%%%%%%%%%%%%%%%%%%%%%%%%%%%%
% Twenty Seconds Resume/CV
% LaTeX Template
% Version 1.1 (8/1/17)
%
% This template has been downloaded from:
% http://www.LaTeXTemplates.com
%
% Original author:
% Carmine Spagnuolo (cspagnuolo@unisa.it) with major modifications by 
% Vel (vel@LaTeXTemplates.com)
%
% License:
% The MIT License (see included LICENSE file)
%
%%%%%%%%%%%%%%%%%%%%%%%%%%%%%%%%%%%%%%%%%

%----------------------------------------------------------------------------------------
%	PACKAGES AND OTHER DOCUMENT CONFIGURATIONS
%----------------------------------------------------------------------------------------

\documentclass[letterpaper]{twentysecondcv} % a4paper for A4

%----------------------------------------------------------------------------------------
%	 PERSONAL INFORMATION
%----------------------------------------------------------------------------------------

% If you don't need one or more of the below, just remove the content leaving the command, e.g. \cvnumberphone{}

\profilepic{avatar.jpg}
\cvname{Sam Robson}
\cvjobtitle{Bioinformatics Lead at\\Centre for Enzyme\\Innovation}
\cvgithub{irunfasterthanmycode.github.io}
\cvaddress{Portsmouth, United Kingdom}
\cvnumberphone{}
\cvmail{}
\cvsite{linkedin.com/in/samrobson/}
% \cvgithub{irunfasterthanmycode.github.io}

%----------------------------------------------------------------------------------------

\begin{document}

%----------------------------------------------------------------------------------------
%	 ABOUT ME
%----------------------------------------------------------------------------------------

\aboutme{Computational biologist with a strong mathematics and statistics background. Extensive experience of maintaining, processing, and analysing Big Data from next generation DNA sequencing. Expertise in a wide variety of data mining, data visualisation, deep learning, and machine learning methods to identify statistically significant trends in high-dimensional data. Bioinformatics Lead at the Centre for Enzyme Innovation and Faculty Bioinformatics Lead at the University of Portsmouth.
} 

%----------------------------------------------------------------------------------------
%	 SKILLS
%----------------------------------------------------------------------------------------

% Skill bar section, each skill must have a value between 0 an 6 (float)
\skills{{C++/2},{TensorFlow/2},{python/3},{MySQL/4},{perl/4},{R/5}}

%------------------------------------------------

% Skill text section, each skill must have a value between 0 an 6
\skillstext{{RStudio/5},{Git/4},{LaTeX/4}}

%----------------------------------------------------------------------------------------

\makeprofile % Print the sidebar

%----------------------------------------------------------------------------------------
%	 EDUCATION
%----------------------------------------------------------------------------------------
\section{Education}

\begin{twenty}
  \twentyitem
    {2004-2008}
    {PhD in \emph{Mathematical Biology and Biophysical Chemistry}}
    {}
    {University of Warwick}
  \twentyitem
    {2003-2004}
    {MSc in \emph{Mathematical Biology and Biophysical Chemistry}}
    {Class: $1^{st}$}
    {University of Warwick}
  \twentyitem
    {1861-1863}
    {BSc in \emph{Mathematics (Hons)}}
    {Class: 2:1}
    {University of Warwick}
\end{twenty}
%----------------------------------------------------------------------------------------


%----------------------------------------------------------------------------------------
%	 EXPERIENCE
%----------------------------------------------------------------------------------------
\section{Experience}

\begin{twenty}
  \twentyitem
    {Since 2017}
    {Centre for Enzyme Innovation, University of Portsmouth}
    {Bioinformatics Lead}
    {Responsible for a small team of bioinformatics researchers to use machine learning algorithms to identify potential novel plastic-degrading enzymes from microbiological genome data}
  \twentyitem
    {2014-2017}
    {Sam Robson Consulting (Self-Employed)}
    {Statistics Consultant}
    {Identified key factors influencing doctor burnout through the use of multivariate mixed-effects regression analysis in the largest study of doctor burnout yet conducted}
\twentyitem
    {2010-2017}
    {The Gurdon Institute, University of Cambridge}
    {Bioinformatician}
    {Developed and maintained pipeline and database for in-depth processing, mining and analysis of high-dimensional genome wide DNA sequencing data}
\twentyitem
    {2008-2010}
    {Wellcome Trust Sanger Institute}
    {Statistical/Mathematical Biologist}
    {Responsible for maintaining, processing and normalizing genome-scale Big Data, including sample QC, concordance analysis with previously published SNP data, data optimization and case-control association testing}    
\end{twenty} 
%----------------------------------------------------------------------------------------


%----------------------------------------------------------------------------------------
%	 ANALYSIS SKILLS
%----------------------------------------------------------------------------------------
\section{Analysis Skills}

Big Data wrangling, maintenance and analysis of extremely large data sets, pipeline development for high-throughput DNA sequencing data, normalization of complex data sets, data visualisation, machine learning, regression analysis (linear and generalised linear models), classification models (\emph{unsupervised}: K-means, hierarchical clustering, mixture models; \emph{supervised}: random forest, K-nearest neighbour, SVM), PCA dimensional reduction \\
%----------------------------------------------------------------------------------------


%----------------------------------------------------------------------------------------
%	 LEADERSHIP SKILLS
%----------------------------------------------------------------------------------------
\section{Leadership Skills}

Bioinformatics Lead and Board Member at the Centre for Enzyme Innovation in charge of a small group of bioinformatics researchers, supervisor for a number of PhD students, Faculty Bioinformatics Lead working with and advising researchers throughout the University on a number of distinct projects, explaining complex technical outputs to non-experts and management \\
%----------------------------------------------------------------------------------------


%----------------------------------------------------------------------------------------
%	 COMMUNICATION SKILLS
%----------------------------------------------------------------------------------------
\section{Communication Skills}

Excellent communication skills at the interface between Life Sciences, able to effectively explain complex analysis concepts to non-specialists, able to maintain extremely high standards when working across a large number of disparate projects, ability to work effectively across disciplines, experience collaborating with industry, excellent ability to identify and solve problems, ability to work on own initiative or as a keen team player, highly motivated \\
%----------------------------------------------------------------------------------------


%----------------------------------------------------------------------------------------
%	 AWARDS
%----------------------------------------------------------------------------------------
\section{Awards}

\begin{twentyshort}
  \twentyitemshort
    {2019}
    {Award of £6 million from Research England E3 Fund}
  \twentyitemshort
    {2019}
    {Award of £5,000 Google Cloud Platform research credits} 
  \twentyitemshort
    {2017}
    {Awarded CStat and CSci Membership of the Royal Statistical Society}
\end{twentyshort}
%----------------------------------------------------------------------------------------


%----------------------------------------------------------------------------------------
%	 SECOND PAGE 
%----------------------------------------------------------------------------------------

\newpage % Start a new page

\makeprofile % Print the sidebar

\section{Publications}

\begin{twentyshort}
  \twentyitemshort{2019}{\emph{Interaction of Sox2 with RNA binding proteins in mouse embryonic stem cells}, Experimental Cell Research. 381, 1, 129-138}
  \twentyitemshort{2019}{\emph{METTL1 Promotes let-7 MicroRNA Processing via m7G Methylation}, Molecular Cell. 74, 6, 1278-1290}
  \twentyitemshort{2019}{\emph{Dystrophic mdx mouse myoblasts exhibit elevated ATP/UTP-evoked metabotropic purinergic responses and alterations in calcium signalling}, Biochimica et Biophysica Acta (BBA) - Molecular Basis of Disease}
  \twentyitemshort{2018}{\emph{SRPK1 maintains acute myeloid leukemia through effects on isoform usage of epigenetic regulators including BRD4}, Nature Communications. 9, 1, 5378}
  \twentyitemshort{2018}{\emph{Phosphorylation of histone H4T80 triggers DNA damage checkpoint recovery}, Molecular Cell. 72, 4, 625-635}
  \twentyitemshort{2018}{\emph{Inhibition of the acetyltransferase NAT10 normalizes progeric and aging cells by rebalancing the Transportin-1 nuclear import pathway}, Science Signaling. 11, 537, eaar5401}
  \twentyitemshort{2017}{\emph{Promoter-bound METTL3 maintains myeloid leukaemia by m6A-dependent translation control}, Nature. 552, 126-131}
  \twentyitemshort{2017}{\emph{A lncRNA fine tunes the dynamics of a cell state transition involving Lin28, let-7 and de novo DNA methylation}, eLife. 6, e23468}
  \twentyitemshort{2017}{\emph{RNA binding by the histone methyltransferases Set1 and Set2}, Molecular and Cellular Biology. 37, 14, e00165-17}
  \twentyitemshort{2016}{\emph{A chemical probe for the ATAD2 bromodomain}, Angewandte Chemie International Edition. 55, 38, 11382-11386}
  \twentyitemshort{2016}{\emph{Discovery of I-BRD9, a selective cell active chemical probe for bromodomain containing protein 9 inhibition}, Journal of Medicinal Chemistry. 59, 4, 1425-1439}
  \twentyitemshort{2015}{\emph{Generation of a selective small molecule inhibitor of the CBP/p300 bromodomain for leukemia therapy}, Cancer Research. 75, 23, 5106-5119}
  \twentyitemshort{2014}{\emph{The breast cancer oncogene EMSY Represses transcription of antimetastatic microRNA miR-31}, Molecular Cell. 53, 5, 806-818}
  \twentyitemshort{2014}{\emph{Recurrent mutations, including NPM1c, activate a BRD4-dependent core transcriptional program in acute myeloid leukemia}, Leukaemia. 28, 2, 311-320}
  \twentyitemshort{2014}{\emph{Glutamine methylation in histone H2A is an RNA-polymerase-I-dedicated modification}, Nature. 505, 7484, 564-568}
  \twentyitemshort{2014}{\emph{BET protein inhibition shows efficacy against JAK2V617F-driven neoplasms}, Leukaemia. 28, 1, 88-97}
  \twentyitemshort{2013}{\emph{The non-coding snRNA 7SK controls transcriptional termination, poising, and bidirectionality in embryonic stem cells}, Genome Biology. 14, 9, R98}
  \twentyitemshort{2012}{\emph{Human RNA Methyltransferase BCDIN3D Regulates MicroRNA Processing}, Cell, 151 (2), 278-288}
  \twentyitemshort{2012}{\emph{Three distinct patterns of histone H3Y41 phosphorylation mark active genes}, Cell Reports. 2, 3, p. 470-477}
  \twentyitemshort{2011}{\emph{Deciphering c-MYC-regulated genes in two distinct tissues}, BMC Genomics. 12, 1, 476}
  \twentyitemshort{2011}{\emph{Inhibition of BET recruitment to chromatin as an effective treatment for MLL-fusion leukaemia}, Nature, 478 (7370), 529-533}
  \twentyitemshort{2010}{\emph{Origins and functional impact of copy number variation in the human genome}, Nature, 464 (7289), 704-712}
  \twentyitemshort{2010}{\emph{Genome-wide association study of copy number variation in 16,000 cases of eight common diseases and 3,000 shared controls},  Nature, 464 (7289), 713-720}
  \twentyitemshort{2010}{\emph{Nucleosome-interacting proteins regulated by DNA and histone methylation}, Cell, 143 (3), 470-484}
  \twentyitemshort{2006}{\emph{c-Myc and downstream targets in the pathogenesis and treatment of cancer}, Recent Patents on Anti-Cancer Drug Discovery. 1, 3, 305-326}
\end{twentyshort}

%----------------------------------------------------------------------------------------

\end{document} 
